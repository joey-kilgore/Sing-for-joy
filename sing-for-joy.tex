\documentclass[10pt,a4paper,oneside,twocolumn]{book}
\usepackage[utf8]{inputenc}
\usepackage{amsmath}
\usepackage{amsfonts}
\usepackage{amssymb}
\usepackage{graphicx}
\usepackage{bibleref}
\usepackage{attrib}
\usepackage{verse}
\usepackage{titlesec}

\titleformat{\chapter}[display]   
{\normalfont\huge\bfseries}{\chaptertitlename\ \thechapter}{20pt}{\Huge}   
\titlespacing*{\chapter}{0pt}{-50pt}{20pt}

\newcommand\Chapter[2]{
  \chapter[#1: {\itshape#2}]{#1\\[2ex]\Large\itshape#2}
}

\def\PreTrib{} \def\PostTrib{} % We'll do our own parens, thanks!

\def\V#1 {\(^{#1}\)}
\def\vattrib#1#2{\attrib{(\bibleverse{#1}#2)}}

\title{Sing for Joy}
\author{Joseph Kilgore}

\begin{document}
\maketitle

\chapter{Sing for Joy}
\section{Introduction}
Music has always been a significant part of culture and worship. While music has progressed from chants, to standardized music writing, to now instantly sending recording around the world, it's ability to affect people has remained. 

Throughout the bible we are given passages describing the importance of shouting and song as part of worship.

\begin{quote}
	\V{1} Come, let us sing for joy to the Lord;\\
    let us shout aloud to the Rock of our salvation.\\
	\V{2} Let us come before him with thanksgiving\\
    and extol him with music and song.
	\vattrib{Psalms}{(95:1-2)}
	
	\medskip
	
	\V{1} Praise the Lord.\\
	Praise God in his sanctuary;\\
    \vin praise him in his mighty heavens.\\
	\V{2} Praise him for his acts of power;\\
    \vin praise him for his surpassing greatness.\\
	\V{3} Praise him with the sounding of the trumpet,\\
    \vin praise him with the harp and lyre,\\
	\V{4} praise him with timbrel and dancing,\\
    \vin praise him with the strings and pipe,\\
	\V{5} praise him with the clash of cymbals,\\
    \vin praise him with resounding cymbals.\\
	\V{6} Let everything that has breath praise the Lord.\\
	Praise the Lord.
	\vattrib{Psalms}{(150:1-6)}
	
	\medskip
	
	\V{19} speaking to one another with psalms, hymns, and songs from the Spirit. Sing and make music from your heart to the Lord, \V{20} always giving thanks to God the Father for everything, in the name of our Lord Jesus Christ.
	\vattrib{Ephesians}{(5:19-20)}
\end{quote}

\section{Notes}
And while music is not the only way to worship (often we treat the two as synonyms), music is certainly a part of worship. We don't need to be musicians to make music for Christ. We don't need to be producers to make music for Christ. We don't even need to sing on key to make music for Christ. All of the music we make in this life will be an imperfect worship. The harps and lyres we use won't be perfectly tuned, the intonation of the guitars and horns will be off, and the tempo of the timbrel and cymbals won't be perfect.

This study is meant to take a look at specifically at the music of our worship. The music has changed throughout history in both timbre and lyrical meaning. We also have to remember that the songs were written by other imperfect humans driven by the Holy Spirit. Hopefully looking through these songs allows us to look at Christ, and our personal relationship with him in different light.

We often say that music or ideas `resonate' with us. The concept of resonating has to do with instruments. When you have two strings of a guitar tuned to the same note, plucking one string will cause the other to start vibrating. The more similar the notes are, the more they vibrate together. That's what happens when we hear ideas that `resonate' with us. We don't know exactly what notes or ideas our heart strings are tuned to, but by plucking other strings we can resonate our heart strings. Throughout this study, look for ideas that resonate with you.

\section{Questions}
\begin{itemize}
\item How significant is music in your life and worship?

\item When you think of worship music, what comes to mind?
\end{itemize}

\Chapter{Amazing Grace}{John Newton}
\section{Background}
John Newton (1725-1807), most well known for writing this hymn, had anything but a straightforward path to Jesus. Having grown up with a shipmaster, Newton grew to join a merchant ship sailing the Mediterranean, and eventually sent into the Royal Navy.

Then, after being rescued following his crew deserting him in West Africa, Newton had his meeting with God. He woke during the voyage back to England to a severe storm. He prayed for God's mercy, and the storm began to die down. From that day on Newton began his conversion to Christianity.

While shortly after accepting the doctrines of the evangelical church (March 10, 1748), he would say his true conversion was later: ``I cannot consider myself to have been a believer in the full sense of the word, until a considerable time afterwards.''

Newton would soon become part of the slave trade, making several voyages until 1754, and even after leaving the seafairing he continued to invest in the slave operations. Though he eventually grew more and more sympathetic for the slaves.

In 1778, he broke his silence on the subject, and began working towards the abolition of slavery. He apologized for ``a confession, which\ldots comes too late\ldots It will always be a subject of humiliating reflection to me, that I was once an active instrument in a business at which my heart now shudders.'' And one year later, \textit{Amazing Grace} was published.

\section{Lyrics}
\begin{flushleft}
Amazing grace! (how sweet the sound)\\
That sav'd a wretch like me!\\
I once was lost, but now am found,\\
Was blind, but now I see.
\medskip

`Twas grace that taught my heart to fear,\\
And grace my fears reliev'd;\\
How precious did that grace appear\\
The hour I first believ'd!
\medskip

Thro' many dangers, toils, and snares,\\
I have already come;\\
`Tis grace hath brought me safe thus far,\\
And grace will lead me home.
\medskip

The Lord has promis'd good to me,\\
His word my hope secures;\\
He will my shield and portion be\\
As long as life endures.
\medskip

Yes, when this flesh and heart shall fail,\\
And mortal life shall cease;\\
I shall possess, within the veil,\\
A life of joy and peace.
\medskip

The earth shall soon dissolve like snow,\\
The sun forbear to shine;\\
But God, who call'd me here below,\\
Will be forever mine.
\end{flushleft}

\section{Verses}
\begin{quote}
	\V{16} Then King David went in and sat before the Lord, and he said:

``Who am I, Lord God, and what is my family, that you have brought me this far? \V{17} And as if this were not enough in your sight, my God, you have spoken about the future of the house of your servant. You, Lord God, have looked on me as though I were the most exalted of men.

\V{18} ``What more can David say to you for honoring your servant? For you know your servant, \V{19} Lord. For the sake of your servant and according to your will, you have done this great thing and made known all these great promises.
  \vattrib{IIChronicles}{(17:16-19)}
  \medskip
  
  \V{12} I thank him who has given me strength, Christ Jesus our Lord, because he judged me faithful, appointing me to his service, \V{13} though formerly I was a blasphemer, persecutor, and insolent opponent. But I received mercy because I had acted ignorantly in unbelief, \V{14} and the grace of our Lord overflowed for me with the faith and love that are in Christ Jesus. \V{15} The saying is trustworthy and deserving of full acceptance, that Christ Jesus came into the world to save sinners, of whom I am the foremost. \V{16} But I received mercy for this reason, that in me, as the foremost, Jesus Christ might display his perfect patience as an example to those who were to believe in him for eternal life. \V{17} To the King of the ages, immortal, invisible, the only God, be honor and glory forever and ever. Amen.
  \vattrib{ITimothy}{(1:12-17)}
\end{quote}

\section{Notes}
Paul and Newton give us a look at how much grace God has. If we need more of a look at how much grace God has, we can look back at what Paul (previously Saul) had done.

\begin{quote}
\V{1} And Saul approved of their killing him.

On that day a great persecution broke out against the church in Jerusalem, and all except the apostles were scattered throughout Judea and Samaria. \V{2} Godly men buried Stephen and mourned deeply for him. \V{3} But Saul began to destroy the church. Going from house to house, he dragged off both men and women and put them in prison.
\vattrib{Acts}{(8:1-3)}
\end{quote}

It is easy for us to look at these men and say how wicked they were. And in doing so, we often then compare their sins to ours. We can confidently say, ``If God can save these wicked men, surely He can save me.'' But that is completely missing the point. Nowhere do we see God saying, ``for all have sinned, but some have sinned more than others.'' Rather we see,

\begin{quote}
\V{21} But now apart from the law the righteousness of God has been made known, to which the Law and the Prophets testify. \V{22} This righteousness is given through faith in Jesus Christ to all who believe. There is no difference between Jew and Gentile, \V{23} for all have sinned and fall short of the glory of God, \V{24} and all are justified freely by his grace through the redemption that came by Christ Jesus.
\vattrib{Romans}{(3:21-24)}
\end{quote}

The slave trader, the killer of God's people, and I are equally broken in the eyes of the Lord. By considering other sin as worse than ours, we are downplaying the severity of our sin. For the wage of \textit{all} sin is death.

\section{Questions}
\begin{itemize}
\item What lyrics or verses stood out to you? Why?

\item Do you find yourself downplaying your own sin?
\end{itemize}


\Chapter{Make My Life a Prayer to You}{Keith Green}
\section{Background}
Keith Gordon Green (October 21, 1953 – July 28, 1982) was an American pianist, singer, and songwriter in the contemporary Christian music genre, who was originally from Sheepshead Bay, Brooklyn, New York. Constantly questioning who or what is God led him to the eastern mysticism movement and drugs of the 70's. After a bad trip, he swore off drugs and began searching for God in the last place left, Jesus. 

Keith strived to preach the True Gospel. His lyrics and performances convicted and transformed. He cared only about preaching the Gospel in the purest form, regardless of whether people would like him for it. His album titles show his motives clearly, \textit{For Him Who Has Ears to Hear}, \textit{No Compromise}, and songs like \textit{Asleep in the Light}, and \textit{Soften Your Heart}.

The Lord took Keith home at 28 in a plane crash, along with his two oldest children, and leaving his then pregnant wife Melody Green and their third child.

\section{Lyrics}
\begin{flushleft}
Make my life a prayer to You\\
I wanna do what you want me to\\
No empty words and no white lies\\
No token prayers no compromise
\medskip 


I wanna shine the light You gave\\
Through Your Son You sent to save us\\
From ourselves and our despair\\
It comforts me to know You're really there
\medskip 

Well I wanna thank you know\\
For being patient with me\\
Oh it's so hard to see\\
When my eyes are on me\\
I guess I'll have to trust\\
And just believe what You say\\
Oh you're coming again\\
Coming to take me away
\medskip 

I wanna die and let You give\\
Your life to me so I might live\\
And share the hope You gave me\\
The love that set me free\\
\medskip 

I wanna tell the world out there\\
You're not some fable or fairy tale\\
That I've made up inside my head\\
You're God the Son and You've risen from the dead
\end{flushleft}

\section{Verses}

\begin{quote}

    \V{1}
    Therefore, I urge you, brothers and sisters, in view of God’s mercy, to offer your bodies as a living sacrifice, holy and pleasing to God—this is your true and proper worship. 
    \V{2}
    Do not conform to the pattern of this world, but be transformed by the renewing of your mind. Then you will be able to test and approve what God’s will is—his good, pleasing and perfect will.
  \vattrib{Romans}{(12:1-2)}

	\medskip

	\V{24} Then Jesus said to his disciples, ``Whoever wants to be my disciple must deny themselves and take up their cross and follow me. 		\V{25} For whoever wants to save their life will lose it, but whoever loses their life for me will find it. 
	\V{26} What good will it be for someone to gain the whole world, yet forfeit their soul? Or what can anyone give in exchange for their soul? 
	\V{27} For the Son of Man is going to come in his Father’s glory with his angels, and then he will reward each person according to what they have done.
	\vattrib{Matthew}{(16:24-27)}
	
	\medskip	
	
	\V{20} I eagerly expect and hope that I will in no way be ashamed, but will have sufficient courage so that now as always Christ will be exalted in my body, whether by life or by death. 
	\V{21} For to me, to live is Christ and to die is gain.
	\vattrib{Philippians}{(1:20-21)}
\end{quote}

\section{Notes}
It is easy for us to not go \textit{all in} for Christ. It is the norm to prioritize your career, family, or even hobbies over Jesus. Often times we try and figure out how to make more time for Jesus in our day. We will say, ``If I just get up 20 minutes earlier I can have more quiet time with God'', or we can say ``I'll start going to another bible study on Thursday nights'', but both are missing the point. God doesn't want a bigger piece of the pie. God wants the whole thing. It's not about making a larger slot in your calendar for Christ. We should strive to make sure that every part of our calendar is working for His glory.

\section{Questions}
\begin{itemize}
\item What lyrics or verses stood out to you? Why?

\item Is there something holding you back from giving Jesus your \textit{entire} life? Not just Sunday morning, Wednesday evenings, Christmas, and Easter, but rather every day. 

\item What is a tangible step you could take that would make more of your life for Christ?
\end{itemize}

\end{document}